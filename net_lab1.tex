\documentclass[a4paper,12pt]{article}
\usepackage{amsmath}
\usepackage{graphicx}
\graphicspath{ {./images/} }



\begin{document}
\title{Where's Fluffymon?}
\date{ 22.03.2021}
\maketitle
\begin{center}
Olariu Alexandra, Tarșa Radu, Nassar Mahmoud, Crăiniceanu Cătălin,  Rădeanu Dragoș
\end{center}
\section{Functional requirements}
Sistemul va permite:
\newline
-utilizatorilor sa se logheze/inregistreze prin introducerea emailului si parolei.\newline
-oricui sa  vada o lista de animale disparute din apropierea locatiei lor.
\newline
-celor care gasesc animale disparute sa posteze mesaje, poze si sa colecteze recompensa la confirmarea stapanilor.
\newline
-utilizatorilor sa adauge comentarii cu detalii care pot ajuta la gasirea animalului.
\newline
-accesarea si pe mobil(aplicatia va fi responsive).
\newline
-utilizatorilor sa vizualizeze/partajeze locatii(GPS)

\section{Nonfunctional requirements}
\emph{Compatibilitate}\newline
Aplicatia va putea fi folosita atat pe mobil cat si pe desktop.
\newline
\emph{Securitate}\newline
Salvarea datelor de inregistrare criptate in baza de date.
\newline
\emph{Disponibilitate}\newline
Aplicatia va fi disponibila 24/7 utilizatorilor logati.
\newline
\emph{Usability}\newline
Aplicatia va fi usor de folosit de toate categoriile de varsta, toate elementele de fundal vor avea dimensiuni mari.
\section{Project structure – github}
\section{Project stack}
\textbf{.NET 5}\newline
\textbf{ Blazor}
\section{ADH}
Pentru a minimiza timpul de incarcare a sectiunii cu anunturi(missing pets), vom incarca un numar finit(10-20) de postari per pagina.\newline

\section{SQ instance}

\end{document}